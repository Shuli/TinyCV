<<<<<<< HEAD
\documentclass{jsarticle}
\usepackage{multicol}
\usepackage{amsmath}
\usepackage{ascmac}

\title{Proof of logical consistency of fog removal by temporal logic}
\author{Hisashi IKARI}
\date{2017/03/06}

\begin{document}

\part{Proof of logical consistency of fog removal by temporal logic}
\label{Proof of logical consistency of fog removal by temporal logic}

The state transition of the calculation process of fog removal based on the Kripke model is as follows[1-4].
The proposition determined based on the Kripke model is satisfied as follows[5-6]. Based on this model, it prevents the use of excessive resources and guarantees parallelism.


\begin{equation}
\begin{split}
M_{mutex}=<S_{mutex}, R_{mutex}, L_{mutex}>
\end{split}
\end{equation}

\begin{equation}
\begin{split}
\begin{gathered}
S_{mutex}=\bigl\{(darkchannel, atmosphere, transmission, guidedfilter,sumfilter,radiance, \\
grayscale, matrix, resource) | darkchannel \in \bigl\{1,...,9\bigr\}, atmosphere \in \bigl\{1,...,13\bigr\}, \\
transmission \in \bigl\{1,...,11\bigr\},guidedfilter \in \bigl\{1,...,45\bigr\}, \\
sumfilter \in \bigl\{1,...,19\bigr\}, radiance \in \bigl\{1,...,12\bigr\}, grayscale \in \bigl\{1\bigr\}, \\
matrix \in \bigl\{1,...,21\bigr\}, resource  \in \bigl\{0, 1\bigr\}\bigr\}
\end{gathered}
\end{split}
\end{equation}

\begin{equation}
\begin{split}
\begin{gathered}
R_{mutex}=\bigl\{(darkchannel, atmosphere, transmission, guidedfilter, sumfilter, radiance, \\
grayscale, matrix, resource), (darkchannel', atmosphere', transmission', \\
guidedfilter',sumfilter', radiance', grayscale', matrix', resource') | \\
(darkchannel, atmosphere, transmission, guidedfilter, sumfilter, \\
radiance, grayscale, matrix, resource) \rightarrow (darkchannel', atmosphere', \\
transmission', guidedfilter', sumfilter', radiance', grayscale', \\
matrix', resource') \in \Sigma \bigr\}
\end{gathered}
\end{split}
\end{equation}

\begin{equation}
\begin{split}
\begin{gathered}
L_{mutex}(darkchannel, atmosphere, transmission, guidedfilter, sumfilter, \\
radiance, grayscale,matrix, resource)=\bigl\{(pc_0=darkchannel), \\
(pc_1=atmosphere),(pc_2=transmission),(pc_3=guidedfilter), \\
(pc_4=sumfilter),(pc_5=radiance),(pc_6=grayscale),(gpu=matrix), (n=resource)\bigr\}
\end{gathered}
\end{split}
\end{equation}

\begin{equation}
\begin{split}
\begin{gathered}
(pc_i=darkchannel) \in L_{mutex}(darkchannel, atmosphere, transmission, \\
guidedfilter, sumfilter,radiance, grayscale, matrix, resource)= \\ 
\bigl\{(pc_0=darkchannel),(pc_1=atmosphere),(pc_2=transmission), \\
(pc_3=guidedfilter),(pc_4=sumfilter),(pc_5=radiance),(pc_6=grayscale),\\
(gpu=matrix), (n=resource)\bigr\} (pc_2=darkchannel) \in L_{mutex}(2, 2, 1)
\end{gathered}
\end{split}
\end{equation}

\begin{equation}
\begin{split}
\begin{gathered}
PA_{mutex}=\bigl\{(pc_i=darkchannel) | i \in\bigl\{0, 1\bigr\}, darkchannel \in \bigl\{1,...,9\bigr\}\bigr\} \\
\cup \bigl\{(gpu=matrix) | resource \in \bigl\{1,...,21\bigr\}\bigr\} \\
\cup \bigl\{(n=resource) | resource \in \bigl\{0, 1\bigr\} \bigr\} \cup \bigl\{false\bigr\}
\end{gathered}
\end{split}
\end{equation}

\end{document}


=======
\documentclass{jsarticle}
\usepackage{multicol}
\usepackage{amsmath}
\usepackage{ascmac}

\title{Proof of logical consistency of fog removal by temporal logic}
\author{Hisashi IKARI}
\date{2017/03/06}

\begin{document}

\part{Proof of logical consistency of fog removal by temporal logic}
\label{Proof of logical consistency of fog removal by temporal logic}

The state transition of the calculation process of fog removal based on the Kripke model is as follows[1-4].
The proposition determined based on the Kripke model is satisfied as follows[5-6]. Based on this model, it prevents the use of excessive resources and guarantees parallelism.


\begin{equation}
\begin{split}
M_{mutex}=<S_{mutex}, R_{mutex}, L_{mutex}>
\end{split}
\end{equation}

\begin{equation}
\begin{split}
\begin{gathered}
S_{mutex}=\bigl\{(darkchannel, atmosphere, transmission, guidedfilter,sumfilter,radiance, \\
grayscale, matrix, resource) | darkchannel \in \bigl\{1,...,9\bigr\}, atmosphere \in \bigl\{1,...,13\bigr\}, \\
transmission \in \bigl\{1,...,11\bigr\},guidedfilter \in \bigl\{1,...,45\bigr\}, \\
sumfilter \in \bigl\{1,...,19\bigr\}, radiance \in \bigl\{1,...,12\bigr\}, grayscale \in \bigl\{1\bigr\}, \\
matrix \in \bigl\{1,...,21\bigr\}, resource  \in \bigl\{0, 1\bigr\}\bigr\}
\end{gathered}
\end{split}
\end{equation}

\begin{equation}
\begin{split}
\begin{gathered}
R_{mutex}=\bigl\{(darkchannel, atmosphere, transmission, guidedfilter, sumfilter, radiance, \\
grayscale, matrix, resource), (darkchannel', atmosphere', transmission', \\
guidedfilter',sumfilter', radiance', grayscale', matrix', resource') | \\
(darkchannel, atmosphere, transmission, guidedfilter, sumfilter, \\
radiance, grayscale, matrix, resource) \rightarrow (darkchannel', atmosphere', \\
transmission', guidedfilter', sumfilter', radiance', grayscale', \\
matrix', resource') \in \Sigma \bigr\}
\end{gathered}
\end{split}
\end{equation}

\begin{equation}
\begin{split}
\begin{gathered}
L_{mutex}(darkchannel, atmosphere, transmission, guidedfilter, sumfilter, \\
radiance, grayscale,matrix, resource)=\bigl\{(pc_0=darkchannel), \\
(pc_1=atmosphere),(pc_2=transmission),(pc_3=guidedfilter), \\
(pc_4=sumfilter),(pc_5=radiance),(pc_6=grayscale),(gpu=matrix), (n=resource)\bigr\}
\end{gathered}
\end{split}
\end{equation}

\begin{equation}
\begin{split}
\begin{gathered}
(pc_i=darkchannel) \in L_{mutex}(darkchannel, atmosphere, transmission, \\
guidedfilter, sumfilter,radiance, grayscale, matrix, resource)= \\ 
\bigl\{(pc_0=darkchannel),(pc_1=atmosphere),(pc_2=transmission), \\
(pc_3=guidedfilter),(pc_4=sumfilter),(pc_5=radiance),(pc_6=grayscale),\\
(gpu=matrix), (n=resource)\bigr\} (pc_2=darkchannel) \in L_{mutex}(2, 2, 1)
\end{gathered}
\end{split}
\end{equation}

\begin{equation}
\begin{split}
\begin{gathered}
PA_{mutex}=\bigl\{(pc_i=darkchannel) | i \in\bigl\{0, 1\bigr\}, darkchannel \in \bigl\{1,...,9\bigr\}\bigr\} \\
\cup \bigl\{(gpu=matrix) | resource \in \bigl\{1,...,21\bigr\}\bigr\} \\
\cup \bigl\{(n=resource) | resource \in \bigl\{0, 1\bigr\} \bigr\} \cup \bigl\{false\bigr\}
\end{gathered}
\end{split}
\end{equation}

\end{document}


>>>>>>> f2525c199a53a1347cfd166a87dd520382da7e56
